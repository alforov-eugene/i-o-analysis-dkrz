\documentclass[]{article}
\usepackage{lmodern}
\usepackage{amssymb,amsmath}
\usepackage{ifxetex,ifluatex}
\usepackage{fixltx2e} % provides \textsubscript
\ifnum 0\ifxetex 1\fi\ifluatex 1\fi=0 % if pdftex
  \usepackage[T1]{fontenc}
  \usepackage[utf8]{inputenc}
\else % if luatex or xelatex
  \ifxetex
    \usepackage{mathspec}
  \else
    \usepackage{fontspec}
  \fi
  \defaultfontfeatures{Ligatures=TeX,Scale=MatchLowercase}
\fi
% use upquote if available, for straight quotes in verbatim environments
\IfFileExists{upquote.sty}{\usepackage{upquote}}{}
% use microtype if available
\IfFileExists{microtype.sty}{%
\usepackage{microtype}
\UseMicrotypeSet[protrusion]{basicmath} % disable protrusion for tt fonts
}{}
\usepackage[unicode=true]{hyperref}
\hypersetup{
            pdftitle={I/O analysis of climate applications},
            pdfauthor={Arne Beer, MN 6489196, Frank Röder, MN 6526113},
            pdfborder={0 0 0},
            breaklinks=true}
\urlstyle{same}  % don't use monospace font for urls
\IfFileExists{parskip.sty}{%
\usepackage{parskip}
}{% else
\setlength{\parindent}{0pt}
\setlength{\parskip}{6pt plus 2pt minus 1pt}
}
\setlength{\emergencystretch}{3em}  % prevent overfull lines
\providecommand{\tightlist}{%
  \setlength{\itemsep}{0pt}\setlength{\parskip}{0pt}}
\setcounter{secnumdepth}{0}
% Redefines (sub)paragraphs to behave more like sections
\ifx\paragraph\undefined\else
\let\oldparagraph\paragraph
\renewcommand{\paragraph}[1]{\oldparagraph{#1}\mbox{}}
\fi
\ifx\subparagraph\undefined\else
\let\oldsubparagraph\subparagraph
\renewcommand{\subparagraph}[1]{\oldsubparagraph{#1}\mbox{}}
\fi

\title{I/O analysis of climate applications}
\author{Arne Beer, MN 6489196, Frank Röder, MN 6526113}
\date{}

\begin{document}
\maketitle

{
\setcounter{tocdepth}{3}
\tableofcontents
}
\pagebreak

\section{Introduction}\label{introduction}

\subsection{About the paper}\label{about-the-paper}

In this paper we analyse and present the pros and cons of different data
structures required by a some carefully picked climate and weather
prediction models. Further we look at the absolute bare minimum of data
required by those models.

\subsection{Getting started}\label{getting-started}

With intent to get an overview about the richness of climate,land,ice,
weather and ocean models we took a look at some in depth to work out
that the approachability and documentation was not that clear. Tons of
very old models passed our way of searching through the sides of got
dusty projects and source code. The question than was to get an good
overview of up to date and easy to handle models which are still
supported and updated.

\subsection{How we want to help}\label{how-we-want-to-help}

bla bla bla jeder kann sich an dem Paper schnell und effektive bedienen

\section{IFS - Integrated Forecasting
System}\label{ifs---integrated-forecasting-system}

\subsection{About IFS}\label{about-ifs}

IFS is a Model by European Centre for Medium-range Weather Forecast
(ECMWF) which is used to make analysis of data. This data can be a
variety of different physical bulks. This model looked quite promising
as they offered an OpenIFS version of the model. After some research we
discovered that the licence forbids ``Commercial and benchmarking use of
OpenIFS models'', which stopped us from further investigation. I would
recommend to use this model in a research or academic context, as there
is plenty of documentation and a big user base.

\section{Unidata - Awips2}\label{unidata---awips2}

AWIPS2 is a package which contains weather forecast display and
analysis. This open-source java application consists of EDEX a data
server and CAVE the client for data analysis and rendering.

\subsection{Installation}\label{installation}

For the installation of awips2 ones can easily download the repository
from github and make it run with installCave.sh and installEDEX.sh.
Those install scripts use yum as a package manager are currently
supported for CentOS, Fedora and RedHead. To make it compatible for the
cluster there is maybe more to be done. Awips2 is normally installed
with the help of the package manger YUM which could lead to some
problems if you' re not the root. Awips2 requires a directory at root
location ``/awips2/''. There are about 2000 lines of code where
``/awips2/'' is hardcoded, so switching directories is not an option.

\textbf{To build} a version for our purpose it would be the best to have
a EDEX on the cluster which is providing our local CAVE with data for
visualisation.

\section{CESM - Community Earth System
Model}\label{cesm---community-earth-system-model}

\subsection{About CESM}\label{about-cesm}

CESM itself consists of seven geophysical models like ocean, land, ice,
atmosphere \ldots{} . The CESM project is made and supported by U.S.
climate reseachers and mainly by the National Science Foundation (NSF).
The scientific development is conducted by the CESM working group twice
a year. For more information related to the development its recommended
to visit the website {[}verlinkung zur quelle{]}.

\subsection{Requirements}\label{requirements}

Here are some preconditions directly taken from the documentation of
CESM. In favour to make it run on the cluster we are working with we
have to walk through the list:

\begin{itemize}
\tightlist
\item
  UNIX style operating system such as CNL, AIX and Linux \checkmark  
\item
  csh, sh, and perl scripting languages \checkmark  
\item
  subversion client version 1.4.2 or greater \checkmark  
\item
  Fortran (2003 recommended, 90 required) and C compilers. pgi, intel,
  and xlf are recommended compilers. \checkmark (gfortran gcc-Version
  4.8)
\item
  MPI (although CESM does not absolutely require it for running on one
  processor) \checkmark
\item
  NetCDF 4.2.0 or newer. \checkmark (Version 7.3 \& 4.2)
\item
  ESMF 5.2.0 or newer (optional).
\item
  pnetcdf 1.2.0 is required and 1.3.1 is recommended ???
\item
  Trilinos may be required for certain configurations X
\item
  LAPACKm or a vendor supplied equivalent may also be required for some
  configurations. \checkmark (Version 3.0)
\item
  CMake 2.8.6 or newer is required for configurations that include CISM.
  \checkmark (Version 2.8.12.2)
\end{itemize}

\subsection{Installation}\label{installation-1}

\begin{itemize}
\item
  Open source
\item
  Download at
  \href{http://www.cesm.ucar.edu/models/cesm1.2/cesm/doc/usersguide/x290.html\#download_ccsm_code}{CCMS(Link)}:

  \begin{itemize}
  \tightlist
  \item
    Username: guestuser
  \item
    Password: friendly
  \end{itemize}
\item
  Version 1.2.1
\item
  Available with svn:

\begin{verbatim}
svn co https://svn-ccsm-models.cgd.ucar.edu/cesm1/release_tags/cesm1_2_1 \
cesm1_2_1 --username guestuser --password friendly
\end{verbatim}
\end{itemize}

Most parts of the CESM software project are open source. However three
libraries are pulbished by the Los Almos National Laboratory, who
licenced their software as free to use as long as it isn't used in a
commercial context. Affected libraries are POP, SCRI and CICE
\href{http://www.cesm.ucar.edu/management/UofCAcopyright.ccsm3.html}{(Link
to licence)}.

\subsection{Input Data Set}\label{input-data-set}

\subsubsection{Setup}\label{setup}

There is actually a set of input data which can be downloaded and
configured for CESM. It can be made available through another Subversion
input data repository by using the same username as used in the
installation above. The dataset is around 1 TByte big and should not be
downloaded at ones. The download is regulated on demand, so if CESM
needs the particullar data it will be downloaded and checked
automatically be CESM itself. The data should be on a disk in the local
area. The CESM variable \texttt{\$DIN\_LOCK\_ROOT} has to be set inside
of the script. Multiple users can use the same
\texttt{\$DIN\_LOCK\_ROOT} directory and should be configurated as group
writeable. If the machine is supported there is a preset otherwise there
is a possibility to make it also run on generic machines with the
varibale argument \texttt{create\_newcase}. Files in the subdirectory of
the \texttt{\$DIN\_LOCK\_ROOT} should be write-protected to exclude
accidentally deleting or changeing of them.

\section{Conclusion}\label{conclusion}

EDEX \& CAVE are supported by the U.S. company Raytheon.

bla bla bla nice project.

\section{References}\label{references}

http://www.cesm.ucar.edu/models/current.html http://www2.cesm.ucar.edu/
http://www.ecmwf.int/en/forecasts/
http://www.unidata.ucar.edu/software/awips2/

\end{document}
